\begin{chapter}{\label{cha:numerics}Numerical Algorithms}
\section{\label{section:RK} Numerical procedures for 2D and 3D solutions}
	\subsection{\label{section:RK4} Fourth order Runge-Kutta scheme}
	The classical fourth-order Runge-Kutta formula (RK4) is described equivalently in many texts. We follow the description in \cite{NumericalRecipes}. Let an initial value problem be specified as
	
	\begin{align*}
		\frac{\partial \psi}{\partial t} &= f(\psi,t),\hspace{0.25in}\psi(t_0) = \psi_0.
	\end{align*}

A step-size, $h>0$, is chosen as the parameter controlling how the solution is advanced over $t$. The scheme for estimating $\psi(t_n)= \psi_n$ is then written
\begin{equation}
\begin{split}
		k_1 &= hf(t_n,\psi_n),\\
		k_2 &= hf(t_n+\frac{h}{2},\psi_n+\frac{k_1}{2}),\\
		k_3 &= hf(t_n+\frac{h}{2},\psi_n+\frac{k_2}{2}),\\
		k_4 &= hf(t_n+h,\psi_n+k_3),\\
		\psi_{n+1} &= \psi_n + \frac{k_1}{6}+ \frac{k_2}{3}+ \frac{k_3}{3} + \frac{k_4}{6} + \mathcal{O}(h^5),\\
		t_{n+1}  &= t_n + h.
		\label{eq:rk4}
\end{split}
\end{equation}


	A derivation of the Runge-Kutta scheme, along with outline proofs of accuracy and stabilty are shown in Appendix \ref{appsection:rk4deriv}.
	In all of our relevant calculations the value of f is set as the right hand side of the homogeneous or trapped GPE. The main loop formulating the RK4 method may be repeated indefinitely to reach any $t>t_0$. The step size for a given set of parameters should be chosen small enough that smaller choices make no quantitative changes to the resulting solution.

	\subsection{\label{section:numericalParams} Numerical stability and convergence}
	We now investigate numerical parameters which affect the stability of simulated superfluid systems. Our direct aim is to find a suitable discretisaton of space and time so that while simulations are timely, our numerical quantities are converged, that is, not overly sensitive to small changes in computational parameters and that quantities conserved by the equations of motion are indeed conserved in the computed numerical solutions.

	We begin by estimating the required distance between our grid points, $\Delta$, in a homogeneous system by considering the width of the vortex core, a feature we would like well and accurately visualised in our numerical solutions. Through observation of a singly quantised vortex core (as in Figure \ref{fig_vortex}) we observe a core radius of approximately $5\xi$ when the background density is $\rho=1$. To ensure the vortex core structure is well resolved we decide to dedicate 10 grid points for a vortex core radius, suggesting a value of $\Delta = \xi/2$.

	\begin{figure}
	\centering
   \begin{tikzpicture}
    \begin{axis}[y tick label style={
		        /pgf/number format/.cd,
		            fixed,
		            fixed zerofill,
		            precision=3,
		        /tikz/.cd
		    },
        width=0.98\linewidth,
        height=0.3\linewidth,
        xlabel={},
        ylabel=$\hat{E}\left ( \hat{t} \right )$,
        xmin=0,
        xmax=10,
        major tick length = 0.07cm
      ]
      \addplot gnuplot [raw gnuplot,mark=none,color=black,thick]{
      	plot "numerics/figures/energ-norm-cons.0.05" using 1:3 with lines;
      };
      \addplot gnuplot [raw gnuplot,mark=none,color=red,thick]{
      	plot "numerics/figures/energ-norm-cons.0.1" using 1:3 with lines;
      };
      \addplot gnuplot [raw gnuplot,mark=none,color=green,thick]{
      	plot "numerics/figures/energ-norm-cons.0.2" using 1:3 with lines;
      };
      \addplot gnuplot [raw gnuplot,mark=none,color=blue,thick]{
      	plot "numerics/figures/energ-norm-cons.0.4" using 1:3 with lines;
      };
    \end{axis}
  \end{tikzpicture}
  \begin{tikzpicture}
    \begin{axis}[y tick label style={
		        /pgf/number format/.cd,
		            fixed,
		            fixed zerofill,
		            precision=4,
		        /tikz/.cd
		    },
        width=0.98\linewidth,
        height=0.35\linewidth,
        name=mainplot,
        xlabel=$\hat{t}$,
        ylabel=$\hat{N}\left ( \hat{t} \right )$,
        xmin=0,
        xmax=10,
        ymax=1.0008,
        major tick length = 0.07cm
      ]
      \addplot gnuplot [raw gnuplot,mark=none,color=black,thick]{
      	plot "numerics/figures/energ-norm-cons.0.05" using 1:4 with lines;
      };
      \addplot gnuplot [raw gnuplot,mark=none,color=red,thick]{
      	plot "numerics/figures/energ-norm-cons.0.1" using 1:4 with lines;
      };
      \addplot gnuplot [raw gnuplot,mark=none,color=green,thick]{
      	plot "numerics/figures/energ-norm-cons.0.2" using 1:4 with lines;
      };
      \addplot gnuplot [raw gnuplot,mark=none,color=blue,thick]{
      	plot "numerics/figures/energ-norm-cons.0.4" using 1:4 with lines;
      };
      \node[anchor=west] (source) at (axis cs:9.7,1.00025){};
      \node[anchor=west] (destination) at (axis cs:9.7,1.0000){};
      \draw[->](source)--(destination);
    \end{axis}
    \begin{axis}[y tick label style={
		        /pgf/number format/.cd,
		            fixed,
		            fixed zerofill,
		            precision=6,
		        /tikz/.cd
		    },
        width=0.4\linewidth,
        height=0.18\linewidth,
        at={(mainplot.north east)},anchor=north east,
        xlabel={},
        ylabel={},
        xmin=9,
        xmax=9.999,
        major tick length = 0.07cm
      ]
      \addplot gnuplot [raw gnuplot,mark=none,color=black,thick]{
      	plot "numerics/figures/energ-norm-cons.0.05" using 1:4 with lines;
      };
      \addplot gnuplot [raw gnuplot,mark=none,color=red,thick]{
      	plot "numerics/figures/energ-norm-cons.0.1" using 1:4 with lines;
      };
      \addplot gnuplot [raw gnuplot,mark=none,color=green,thick]{
      	plot "numerics/figures/energ-norm-cons.0.2" using 1:4 with lines;
      };
    \end{axis}
  \end{tikzpicture}
  \caption{Dimensionless energy and system norm throughout numerical propagation of a trapped condensate with interaction energy $\hat{g}=2000$, chemical potential $\hat{\mu}=25.27$. (Inset) Zoomed view of the convergence of $\hat{N}$ in the system norm. The numerical grid width varies with each line; (black) $\Delta = 0.05$, (red) $\Delta = 0.1$, (green) $\Delta = 0.2$ and (blue) $\Delta = 0.4$.}\label{fig_energ_norm_cons}
 \end{figure}

	In the trapped case we can use the same idea. It is shown in Section \ref{section:healing} that $\xi = \hbar/\sqrt{mg}$ for $\rho=1$. We can then easily rearrange to find an expression for $\xi$ in the harmonic oscillator units of trapped condensates. We find that our approximate grid spacing to adequately resolve the vortex core is $\Delta = 0.5\xi = 0.5\omega \sqrt{\hbar/(\mu \omega)} l_r$. As an example, for a trapped condensate with interaction energy $\hat{g}=2000$, chemical potential $\hat{\mu}=25.27$ and trap frequency $\omega=8.75~{\rm Hz}$ we find that a value of $\Delta=0.1l_r$ should be adequate. 

	In addition to this process, for each set of simulation parameters it is recommended to confirm the suitability of the chosen $\Delta$ by testing the convergence and conservation in the numerical methods. The total condensate energy and norm are good measures for this as they should both be well conserved by the GPE with a dissipation of $\gamma=0$. An example of this process is shown in Figure \ref{fig_energ_norm_cons}: For a large $\Delta>0.1$ both the condensate energy and norm fluctuate wildly. For $\Delta=0.1$ the norm is well extremely well conserved to within $2.10^{-5}\%$ and the energy is conserved within $5.10^{-3}\%$. Observing the smallest tested values with $\Delta<0.1$ we also confirm the the value for the total energy has converged. We conclude that in this case for the chosen parameters there is little reason to use $\Delta<0.1$, the value suggested by the above analysis.

\section{\label{section:vortexidentifying} Identifying vortices}


\subsection{\label{section:gaussianblur} Image filters and the Gaussian kernel}

\section{\label{section:vortexclustering} Quantifying vortex clustering}
	\subsection{\label{section:reevesalgorithm} Recursive Cluster Algorithm (RCA) }
		


	\subsection{\label{section:ripleysk} Ripley's K function }
		\begin{equation}\label{eq:ripleysk}
		K(x) = \frac{A}{n^2}\sum\limits_{i \ne j} I\left (d_{ij}<x\right ),
		\end{equation}
		where $d_{ij}$ is the distance between the $i$th and $j$th points, $A$ is the area of the region containing every point, $n$ is the number of points, $x$ is the search radius, and I is the indicator function (1 if its argument is true, 0 otherwise). Should the points be distributed homogeneously in space, then $K(s)\approx\pi s^2$.
\section{\label{section:vortextracking} Tracking vortex trajectories}
\section{\label{section:vortexremoval} Removing vortices with phase unwrapping}

\section{\label{section:quasi-condensate} Quasi-Condensate Visualisation}
	

\end{chapter}
