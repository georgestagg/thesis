\begin{chapter}{\label{cha:conc}Conclusions and future work}
\section{Conclusions}
In this chapter, we summarise the conclusions of the work presented throughout the thesis, and suggest relevant directions in which the work can be extended in the future. 

In Part I we introduced the mean-field theory that allows for accurate modelling of a dilute, weakly interacting Bose-Einstein condensate. We described the GPE, a non-linear Schr\"odinger equation used to model condensates at zero temperature, along with extensions to model finite temperatures through phenomenological damping and the classical-field method. We also described the theory and practice of various numerical procedures we have used throughout the thesis.

\subsubsection{Classical-like wakes behind elliptical obstacles in Bose-Einstein condensates}
In Chapter \ref{cha:wake} we showed that a 2D or 3D obstacle in the presence of superfluid flow in a Bose-Einstein condensate generates wakes of quantum vortices which resemble those of classical viscous flow past a cylinder or sphere.

We demonstrated a key ingredient to produce classical-like wakes: high vortex nucleation rate so that vortices undergo strong interactions with their neighbours, rather than being swept away. The role of ellipticity in this chapter was to reduce the critical velocity for vortex nucleation and increase vortex nucleation frequency and density, to facilitate strong interaction between vortices.  

The symmetric wakes produced are similar to those observed in classical flow at low $\Rey$. We showed that they are unstable, forming time-dependent asymmetric structures similar to the B\'enard--von K\'arm\'an vortex street of classical fluid dynamics.

The effects we described are relevant to the motion of objects such as vibrating wires, grids and forks in superfluid helium, where the obstacle's ellipticity plays a role which is analogous to rough boundaries \cite{blaz08,brad05}, and described patterns in the density distribution that could be potentially observed in experimental atomic Bose-Einstein condensates, with moving laser-induced potentials.

\subsubsection{Decay of 2D quantum turbulence in a highly oblate Bose-Einstein condensate}
In Chapter \ref{cha:shin} , we modelled the experimental set up of Kwon {\it et al.} in which the creation and decay of vortices within a BEC~\citep{kwon_moon_14} was observed. We showed that the system is well described by simulations of the 2D-GPE with phenomenological dissipation (despite the system's technically 3D nature). We elucidated the system's experimentally unobserved early stages, showing that vortex clusters form behind a laser-induced obstacle. We demonstrated that early time symmetry breaking causes disorganisation of the vortices and that by the time the obstacle is removed, the vortices are well randomised. We confirmed the occasional appearance of crescent-shaped density features, resulting either from the proximity of vortex cores or from a sound pulse which follows a vortex-antivortex reconnection.

We showed that the vortices decay in a manner which is consistent with the two mechanisms proposed by Kwon {\it et al.} (loss of vortices at the condensate edge due to thermal dissipation and vortex-antivortex annihilation events within the condensate) and fitted the rate equations proposed by Kwon {\it et al.} and Cidrim {\it et al.} to the vortex decay that we observed in our numerical simulations. We concluded that Cidrim's equation fits to the data most favourably, while providing physically realistic values for the decay rates.

\subsubsection{Quasi-classical turbulence and the critical velocity in a quenched Bose gas}
In Chapter \ref{cha:nonequib}, we modelled a finite temperature homogeneous Bose gas. We evolved the classical field from highly non-equilibrium initial conditions, through decay of a vortex tangle, to thermalised equilibrium states with a range of temperatures and condensate fractions.

We characterised the turbulent vortex tangle by finding a kinetic energy spectrum that demonstrates a {\it lack} of quasi-classical turbulence, and through tracking the vortex line-density over time, we found a decay rate characteristic of ultra-quantum turbulence. We confirmed the result by calculating the velocity correlation function and integral scale, finding them to be consistent with ultra-quantum turbulence.

With the resulting equilibrium states we inserted a cylindrical obstacle with Gaussian profile, and imposed a fluid flow.  We found that above the critical velocity, vortices are nucleated as wiggly vortex lines, vortex rings, or as a vortex tangle. We demonstrated that the critical velocity decreases with increasing temperature (becoming zero at the critical temperature for condensation) and scales with the square root of the condensate fraction.

\subsubsection{Simulating the rough surface of a ``Floppy Wire''}
In Chapter \ref{cha:afm}, we modelled the rough surface of a real ``floppy wire'' used in helium II experiments through a potential term in the zero temperature GPE and by imposing a flow. The surface of the wire was provided via atomic force microscopy. We performed two-dimensional simulations of the surface at various levels of truncation and at various flow speeds to probe the parameter space.

We observed for various truncations of the 2D profile and for several flow velocities in excess of the critical velocity, the formation of a `boundary layer' of quantum vortices. In each case, for a high enough imposed flow velocity the boundary layer effect breaks down and quantum vortices instead fill the computational box.

We performed large scale 3D simulations, showing that the boundary layer effect generalises to three dimensions. We showed evidence of a ``vortex mill'' mechanism leading to escaping vortex rings formed from the boundary layer. We measured the velocity profile of the fluid flow and demonstrated a qualitatively similar profile to those seen for boundary layers in classical viscous fluids.

Our results suggest that in superfluids the surfaces of moving objects may be covered by a thin `superfluid boundary layer' consisting of vortex loops and rings. This is a surprising effect: boundary layers usually arise from viscous forces, which in zero temperature superfluids are completely absent.

\section{Future work} 
\subsubsection{Other quantum analogues of classical-like wakes}
Many studies of viscous flow in classical fluids have been performed over the years. Collections such as Van Dyke's {\it Album of Fluid Motion} \cite{nagib} demonstrate the wide range of flows possible with various obstacle shapes and sizes in both 2D and 3D regimes. As an example, the wakes of rectangular (rather than circular) obstacles and recesses are shown in Figure \ref{fig:dyke-imgs}. Chapter \ref{cha:wake} investigated the classical-like wakes in the simplest case of a cylinder in quantum flow. It would be an extremely interesting direction for future work to attempt to experiment with the more complicated examples of classical flow patterns that exist in the literature. The idea that the behaviour of many quantum vortices collectively reproduces classical physics would suggest that perhaps other analogies of classical fluid wakes exist in the quantum fluid realm. 
\begin{figure}
\centering
    \includegraphics[width=\linewidth]{wake/square.png}
  \caption{Various examples of classical viscous fluid flow around square obstacles and recesses \cite{nagib}.} 
  \label{fig:dyke-imgs}
\end{figure}

\subsubsection{Finite temperature trapped Bose gas}
While the work set our in Chapter \ref{cha:nonequib} is based on a homogeneous system, in reality Bose-Einstein condensates are experimentally confined in traps, rendering the gas inhomogeneous. One can expect significant corrections to the critical velocity due to density gradients, as well as modifications to the vortex nucleation pattern. It would be interesting to see these higher order effects studied in future works, in particular in the common case of a harmonic trapping potential.

On the other hand, recent advances have led to the formation of quasi-homogeneous condensates in box-like traps \cite{gaunt_2013,chomaz_2015}. Here, the higher order corrections associated with trapping inhomogeneity should have minimal effect. An interesting direction for future experimental work would be to compare our experimentally measurable numerical predictions, such as critical velocity or the vortex line-density decay rate, to experimental data incorporating quasi-homogeneous box-like traps.

\subsubsection{Exploring the superfluid boundary layer}
The experimental implications of possible `superfluid boundary layers`, demonstrated in Chapter \ref{cha:afm}, on macroscopic observables need to be investigated.  The results we have shown in 3D should particularly stimulate experiments in $^3$He-B, where, due to relative
large healing length, it is possible to observe vortex core structure and study flows with controlled surface height roughness.

An alternative experimental application of the 2D boundary layers observed in Chapter \ref{cha:afm} can be found in the context of atomic BECs. Recently, the path has opened towards the realisation of dynamic and arbitrarily shaped obstacles \cite{Henderson09}. By utilizing arbitrary potentials, a moving rough surface could be implemented in a quasi-two-dimensional atomic BEC.

Numerically there is also much work to still be done, particularly in 3D. While current computational limits are reached quickly with our simulations, the potential future increase in power of high performance clusters or even desktop computers could allow for quicker numerical simulation and easier visualisation of results. Along these lines, a potential extension to the work in Chapter \ref{cha:afm} is the undertaking of a significant exploration of the 3D parameter space. It would be interesting to explore the response to changes to the imposed superfluid flow speed and surface roughness through truncation, and if the behaviour generalises from our two-dimensional results to the three-dimensional case.
\end{chapter}