\begin{chapter}{\label{cha:theoretical_model}Theoretical Modeling of BEC}
\section{\label{section:meanfield} Mean-field description}
\section{\label{section:gpe} The Gross-Pitaevskii Equation}
	\begin{equation}
	\mathrm{i} \hbar \frac{\partial\Psi({\bf r},t)}{\partial t} = \left(-\frac{\hbar^2}{2m}\nabla^2 + V({\bf r},t) + g|\Psi({\bf r},t)|^2 - \mu \right) \Psi({\bf r},t).
	\label{eq:gpe}
	\end{equation} 
	Where $V({\bf r},t) = V_{\mathrm{obj}}({\bf r},t) + V_{\mathrm{trap}}({\bf r},t)$. When trapped, the trap is harmonic and of the form $V_{\mathrm{trap}}=m\omega^2r^2/2$.
\section{\label{section:gpedimless} Dimensionless Gross-Pitaevskii Equations}
	\subsection{\label{section:gpedimlesshomg} Homogeneous GPE}
		When discussing a homogeneous condensate we drop the dimensionless modifiers for each quantity and use the equation,
		\begin{equation}\label{eq:dimgpehomg}
		\mathrm{i}\frac{\partial\psi({\bf r},t)}{\partial t} = \left( -\frac{1}{2}\nabla^2 + |\psi({\bf r},t)|^2 + V_{\mathrm{obj}}({\bf r},t) - 1 \right) \psi({\bf r},t).
		\end{equation}
	\subsection{\label{section:gpedimlesstrap} Trapped GPE}
		When discussing the a trapped condensate we drop the dimensionless modifiers for each quantity and use the equation,
		\begin{equation}\label{eq:dimgpetrapped}
		\mathrm{i}\frac{\partial\phi({\bf r},t)}{\partial t} = \left( -\frac{1}{2}\nabla^2 + g|\phi({\bf r},t)|^2 + V({\bf r},t) - 1 \right) \phi({\bf r},t).
		\end{equation}
		Where $V({\bf r},t) = \frac{r^2}{2} + V_{\mathrm{trap}}({\bf r},t)$
\section{\label{section:gpe} The Dissipative Gross-Pitaevskii Equation}
	\subsection{\label{section:gamma} Phenomenological dissipation}
	In the case with a homogeneous condensate this leaves us with
		\begin{equation}\label{eq:dissgpehomg}
		(\mathrm{i} - \gamma)\frac{\partial\psi({\bf r},t)}{\partial t} = \left( -\frac{1}{2}\nabla^2 + |\psi({\bf r},t)|^2 + V_{\mathrm{obj}}({\bf r},t) - 1 \right) \psi({\bf r},t),
		\end{equation}
	and in the case with a trapped condensate this leaves us with
		\begin{equation}\label{eq:dissgpetrapped}
		(\mathrm{i}-\gamma)\frac{\partial\phi({\bf r},t)}{\partial t} = \left( -\frac{1}{2}\nabla^2 + g|\phi({\bf r},t)|^2 + V({\bf r},t) - 1 \right) \phi({\bf r},t).
		\end{equation}

	\subsection{\label{section:mu} The role of the chemical potential}
\section{\label{section:hydrodynamic} Hydrodynamic interpretation}
	We can perform the Madelung transformation $\Psi({\bf r},t) = \sqrt{n({\bf r},t)}\exp (\mathrm{i}\Phi({\bf r},t))$.
	[TODO: do this derivation(yellow booklet)]
	gives us a continuity equation
	\begin{equation}
		\frac{\partial}{\partial t} n + \nabla(n{\bf v}) = 0
		\label{eq:MTcont}
	\end{equation}
	and an Euler-like equation:
	\begin{equation}
		m\frac{\partial}{\partial t} v + \nabla\left(V + gn + \frac{mv^2}{2} - \frac{\hbar^2}{2m\sqrt{n}}\nabla\sqrt{n}\right) = 0
		\label{eq:MTeuler}
	\end{equation}
	$\mathbf{v} = \hbar/m\nabla\Phi \Rightarrow$ Flow is irrotational, except when there are singularities in the phase. 
\section{\label{section:solutions} A selection of analytical solutions}
	\subsection{\label{section:wall} Density near a wall}
	\subsection{\label{section:soliton} Soliton solutions}
	\subsection{\label{section:vortices} Vortex solutions}
\section{\label{section:inital} Initial Conditions}
	\subsection{\label{section:tftrap} Thomas Fermi profile of a trapped condensate}
	\begin{equation}
	\sqrt{\frac{\mu - V({\bf r},t)}{g}} =  \Psi({\bf r},t)
	\label{eq:TF}
	\end{equation} 
	\subsection{\label{section:cfield} Classical Field approximation with a homogeneous condensate}
		\begin{equation}
		\psi({\bf r},t) = \sum_{\bf k} a_{\bf k} \exp (\mathrm{i}{\bf k}\cdot{\bf r}),
		\label{eq:cFieldIC}
		\end{equation}
		where the complex Fourier amplitudes $a_{\bf k}$ are related to the occupation numbers $n_{\bf k}$ through $\braket{a_{\bf k}^{\,}a_{\bf k'}^*} = n_{\bf k}\updelta_{\bf kk'}$. The phase of the complex amplitudes $a_{\bf k}$ are distributed uniformly on $[0,2\pi]$ while $|a_{\bf k}|$ is distributed randomly with fixed mean equal to unity; it has been found that different distributions of $|a_{\bf k}|$ make no qualitative difference to the turbulent evolution[phys rev A 66 013603]. [TODO: write about choice of E and N here to get different condensate fractions]

		We also have the integral distribution function,
		\begin{equation}
		D_k = \sum_{k'<k}n_{\bf k},
		\label{eq:intDistFunc}
		\end{equation}
		a coarse-grained characteristic of the particle distribution which shows how many particles have momenta less than $k$.
\end{chapter}