\begin{chapter}{Detailed Derivations\label{app:App2}}
\section{\label{appsection:madtrans} Derivation of the Hydrodynamic Equations via the Madelung Transformation}
Inserting the Madelung transformation (Section \ref{section:hydrodynamic}) into the GPE and writing the result in tensor notation yields
\begin{equation*}
  \mathrm{i}\hbar\left( \frac{\partial R}{\partial t} + \mathrm{i}\frac{\partial \theta}{\partial t} R \right)e^{\mathrm{i}\theta} =
  -\frac{\hbar^2}{2m}e^{\mathrm{i}\theta}\left( \frac{\partial^2 R}{\partial x_j^2} + 2\mathrm{i}\frac{\partial \theta}{\partial x_j}\frac{\partial R}{\partial x_j}+
  \mathrm{i}\frac{\partial^2 \theta}{\partial x_j^2}R -  \frac{\partial \theta}{\partial x_j}\frac{\partial \theta}{\partial x_j} R  \right) + gR^3e^{\mathrm{i}\theta} + VRe^{\mathrm{i}\theta}.
\end{equation*}
The real and imaginary parts of the GPE, once divided by $\exp (\mathrm{i}\theta)$, then take the form
\begin{align}
  -\hbar R \frac{\partial \theta}{\partial t} &= -\frac{\hbar^2}{2m}\left( \frac{\partial^2 R}{\partial x_j \partial x_j} - R \frac{\partial \theta}{\partial x_j}\frac{\partial \theta}{\partial x_j}  \right) + gR^3 + VR, \label{eq:MTre}\\
  \hbar \frac{\partial R}{\partial t} &= -\frac{\hbar^2}{2m}\left( 2\frac{\partial \theta}{\partial x_j}\frac{\partial R}{\partial x_j} + R \frac{\partial^2 \theta}{\partial x_j \partial x_j} \right).
  \label{eq:MTim}
\end{align}
Consider Equation (\ref{eq:MTim}) and note that $\rho = mR^2 \Rightarrow \frac{\partial \rho}{\partial t} = 2mR\frac{\partial R}{\partial t}$, allowing us to rewrite the equation in terms of $\rho$,
\begin{align*}
  \frac{\partial \rho}{\partial t} &= -\hbar R\left( 2 \frac{\partial \theta}{\partial x_j} \frac{\partial R}{\partial x_j} + R \frac{\partial^2 \theta}{\partial x_j\partial x_j} \right)\\
  &= -2mR\frac{\partial R}{\partial x_j}\frac{\partial}{\partial x_j}\left( \frac{\hbar}{m} \theta \right) - mR^2 \frac{\partial^2}{\partial x_j \partial x_j}\left(\frac{\hbar}{m}\theta \right)\\
  &= -\frac{\partial \rho}{\partial x_j}\frac{\partial}{\partial x_j}\left( \frac{\hbar}{m} \theta \right) - \rho \frac{\partial^2}{\partial x_j \partial x_j}\left(\frac{\hbar}{m}\theta \right).
\end{align*}
The terms containing the phase can then be directly replaced with the fluid velocity, $v_j = \frac{\partial}{\partial x_j}\left( \frac{\hbar}{m} \theta \right)$.
\begin{align*}
  \frac{\partial \rho}{\partial t} &= -\frac{\partial \rho}{\partial x_j} v_j - \rho \frac{\partial}{\partial x_j} v_j\\
                   &= -\frac{\partial}{\partial x_j} \left( \rho v_j \right).
\end{align*}
Rewritten in vector form the result is a continuity equation,
\begin{equation}
  \frac{\partial \rho}{\partial t} + \nabla(\rho{\bf v}) = 0.
  \label{eq:MTcont}
\end{equation}
Now consider Equation (\ref{eq:MTre}), written in the form
\begin{equation*}
\frac{\hbar}{m} \frac{\partial \theta}{\partial t} = \frac{\hbar^2}{2m^2} \left( \frac{1}{R} \frac{\partial^2 R}{\partial x_j \partial x_j} - \frac{\partial \theta}{\partial x_j}\frac{\partial \theta}{\partial x_j}  \right) - \frac{gR^2}{m} - \frac{V}{m}.
\end{equation*}
Note that it can easily be shown $\frac{1}{R} \frac{\partial^2 R}{\partial x_j \partial x_j} = \frac{1}{\sqrt{\rho}}\nabla^2\sqrt{\rho}$ and $\frac{\hbar^2}{2m^2} \frac{\partial \theta}{\partial x_j}\frac{\partial \theta}{\partial x_j} = \frac{v^2}{2} $. It follows that Equation (\ref{eq:MTre}) can be written,

\begin{align*}
&\frac{\hbar}{m} \frac{\partial \theta}{\partial t} = \frac{\hbar^2}{2m^2} \frac{1}{\sqrt{\rho}} \nabla^2\sqrt{\rho} - \frac{v^2}{2} - \frac{gR^2}{m} - \frac{V}{m}\\
&\Rightarrow \frac{\partial}{\partial t}\left(\frac{\hbar}{m} \frac{\partial \theta}{\partial x_k}\right) = \frac{\partial}{\partial x_k}\left(\frac{\hbar^2}{2m^2} \frac{1}{\sqrt{\rho}} \nabla^2\sqrt{\rho} \right)- \frac{\partial}{\partial x_k} \left (\frac{v^2}{2}\right) - \frac{2gR}{m}\frac{\partial R}{\partial x_k} - \frac{1}{m}\frac{\partial V}{\partial x_k}\\
&\Rightarrow \rho\frac{\partial v_k}{\partial t} =\rho \frac{\partial}{\partial x_k}\left(\frac{\hbar^2}{2m^2} \frac{1}{\sqrt{\rho}} \nabla^2\sqrt{\rho} \right)- \rho\frac{\partial}{\partial x_k} \left (\frac{v^2}{2}\right) - 2gR^3\frac{\partial R}{\partial x_k} - \rho\frac{1}{m}\frac{\partial V}{\partial x_k}.
\end{align*}

By noticing that $p = \frac{1}{2}g \left ( \frac{\rho}{m} \right ) ^2 = \frac{gR^4}{2}$ we can write $\frac{\partial p}{\partial x_k} = 2gR^3\frac{\partial R}{\partial x_k}$ and then,
\begin{equation*}
\rho\frac{\partial v_k}{\partial t} + \rho\frac{\partial}{\partial x_k} \left (\frac{v^2}{2}\right) =\rho \frac{\partial}{\partial x_k}\left(\frac{\hbar^2}{2m^2} \frac{1}{\sqrt{\rho}} \nabla^2\sqrt{\rho} \right) - \frac{\partial p}{\partial x_k} - mR^2\frac{\partial}{\partial x_k}\left ( \frac{V}{m}\right ).
\end{equation*}

We now now use the following two results,
\begin{align*}
v_j \frac{\partial}{\partial x_j}v_k &= \frac{\partial}{\partial x_k}\left ( \frac{v_jv_j}{2}\right )\\
2\frac{\partial}{\partial x_k}\left( \frac{1}{\sqrt{\rho}} \frac{\partial^2}{\partial x_j \partial x_j} \sqrt{\rho}\right) &= \frac{1}{\rho} \frac{\partial}{\partial x_j}\rho \frac{\partial}{\partial x_j}\frac{\partial}{\partial x_k} \ln{\rho},
\end{align*}

and find,
\begin{equation*}
\rho\left ( \frac{\partial}{\partial t} v_k v_j\frac{\partial v_k}{\partial x_j}\right) = -\frac{\partial p}{\partial x_k} - \frac{\partial}{\partial x_j} P_{jk} - \rho \frac{\partial}{\partial x_k}\left( \frac{V}{m} \right),
\end{equation*}
where $P_{jk} = -\frac{\hbar^2}{4m^2}\rho\frac{\partial^2\ln{\rho}}{\partial x_j \partial x_k}$. Writing this in vector notation, we obtain an equation similar to the Euler equation for an inviscid fluid,
\begin{equation}
\rho\left( \frac{\partial \mathbf{v}}{\partial t} + \left( \mathbf{v} \cdot \nabla \right)\mathbf{v} \right) = -\nabla p - \nabla \mathbf{P} - \rho \nabla \left(\frac{V}{m}\right).
\end{equation}

\section{\label{appsection:gpeqft} Derivation of the Gross-Pitaevskii Equation through Quantum Field Theory}
This section derives the GPE following the methodology outlined in [NPP GPE tutorial]. We begin by revisting the quantum field theory formalism used to describe a many body quantum system[Fetter 71]. Such a system is described by an N-body wavefunction, $\tilde{\bm\upPsi}(\mathbf{r}_1...\mathbf{r}_N,t)$ which obeys the famous Schr\"odinger equation
\begin{equation}
i \hbar\frac{\partial}{\partial t}\tilde{\bm\upPsi}(\mathbf{r}_1...\mathbf{r}_N,t) = \hat{H}\tilde{\bm\upPsi}(\mathbf{r}_1...\mathbf{r}_N,t),
\end{equation}
where $\mathbf{r}_i$ describes the coordinates of the $i$th body. Now consider a closed system containing a dilute, weakly interacting Bose gas of N atoms. Such a system would be described by $\tilde{\bm\upPsi}(\mathbf{r}_1...\mathbf{r}_N,t)$, with a Hamiltonian of the form 
\begin{equation}
\hat{H} = \sum_{k=1}^N\hat{h}_0(\mathbf{r}_k) + \frac{1}{2}\sum_{k,l=1}^N \hat{V}(\mathbf{r}_k,\mathbf{r}_l).
\end{equation}
Here $\hat{h}_0(\mathbf{r}_k) = -\frac{\hbar}{2m}\nabla^2+V_{ext}(\mathbf{r}_k,t)$ is a contribution arising from the effects of a single particle in an external potential. We assume in the dilute gas all interactions are binary; and so the second term arises from collisions between 2 atoms only. The factor of $\frac{1}{2}$ ensures the effects are only counted once over the entire sum.

We now reformulate this system in a different representation, using the so called `occupation number' orthonormal basis $\ket{n_1...n_\infty}$. This basis arises from the observation that multiple particles sharing an energetically accessible state are indistinguishable. Instead we consider only the number of particles in each state $i$ and denote this $n_i$. Such states often correspond to states with fixed energy $\varepsilon_i$. While the number of states are infinite, our system contains a fixed number of bosons, $N$, implying that there are at most N states occupied.

The wavefunction is mapped into the `occupation number' basis via
\begin{equation}
\tilde{\bm\upPsi}(\mathbf{r}_1...\mathbf{r}_N,t) \rightarrow \ket{\tilde{\bm\upPsi}(t)}=\sum_{n_1...n_\infty} c(n_1...n_\infty)\ket{n_1...n_\infty},
\end{equation}
with $c(n_1...n_\infty)$ appropriately chosen complex coefficients. The values $c$ must follow the particle statistics rules (i.e for Bosons much be symmetric under swapping of quantum numbers) and be normalised so that the probabilities correctly sum to one,

\begin{equation}
\int|\tilde{\bm\upPsi}|^2~d\mathbf{r}=1 \Rightarrow \sum_{n_1...n_\infty}|c(n_1...n_\infty)|^2\frac{N!}{n_1!...n_\infty!} = 1
\end{equation}

\section{\label{appsection:energy} Energy}
\section{\label{appsection:force} Force}

\begin{equation}
T_{jk} = \rho v_j v_k + p\mathrm{\delta}_{jk} - \frac{\hbar^2}{4m^2}\rho \frac{\partial}{\partial x_j}\frac{\partial}{\partial x_k} \ln \rho
\label{eq:stressTensor}
\end{equation}

\begin{equation}
F_k =\frac{\partial}{\partial t} \int_V J_k\,\mathrm{d}V = - \int_V \frac{\partial}{\partial x_j} T_{jk}\,\mathrm{d}V - \int_V \rho \frac{\partial}{\partial x_k} \left( \frac{V}{m} \right)\,\mathrm{d}V
\label{eq:force}
\end{equation}


\end{chapter}
